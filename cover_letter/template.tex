% Options for packages loaded elsewhere
% Options for packages loaded elsewhere
\PassOptionsToPackage{unicode}{hyperref}
\PassOptionsToPackage{hyphens}{url}
\PassOptionsToPackage{dvipsnames,svgnames,x11names}{xcolor}
%
\documentclass[
  12pt,
  letterpaper]{letter}
\usepackage{xcolor}
\usepackage[margin=1in,bottom=1in,left=1in,right=1in]{geometry}
\usepackage{amsmath,amssymb}
\setcounter{secnumdepth}{-\maxdimen} % remove section numbering
\usepackage{iftex}
\ifPDFTeX
  \usepackage[T1]{fontenc}
  \usepackage[utf8]{inputenc}
  \usepackage{textcomp} % provide euro and other symbols
\else % if luatex or xetex
  \usepackage{unicode-math} % this also loads fontspec
  \defaultfontfeatures{Scale=MatchLowercase}
  \defaultfontfeatures[\rmfamily]{Ligatures=TeX,Scale=1}
\fi
\usepackage{lmodern}
\ifPDFTeX\else
  % xetex/luatex font selection
\fi
% Use upquote if available, for straight quotes in verbatim environments
\IfFileExists{upquote.sty}{\usepackage{upquote}}{}
\IfFileExists{microtype.sty}{% use microtype if available
  \usepackage[]{microtype}
  \UseMicrotypeSet[protrusion]{basicmath} % disable protrusion for tt fonts
}{}
\makeatletter
\@ifundefined{KOMAClassName}{% if non-KOMA class
  \IfFileExists{parskip.sty}{%
    \usepackage{parskip}
  }{% else
    \setlength{\parindent}{0pt}
    \setlength{\parskip}{6pt plus 2pt minus 1pt}}
}{% if KOMA class
  \KOMAoptions{parskip=half}}
\makeatother
% Make \paragraph and \subparagraph free-standing
\makeatletter
\ifx\paragraph\undefined\else
  \let\oldparagraph\paragraph
  \renewcommand{\paragraph}{
    \@ifstar
      \xxxParagraphStar
      \xxxParagraphNoStar
  }
  \newcommand{\xxxParagraphStar}[1]{\oldparagraph*{#1}\mbox{}}
  \newcommand{\xxxParagraphNoStar}[1]{\oldparagraph{#1}\mbox{}}
\fi
\ifx\subparagraph\undefined\else
  \let\oldsubparagraph\subparagraph
  \renewcommand{\subparagraph}{
    \@ifstar
      \xxxSubParagraphStar
      \xxxSubParagraphNoStar
  }
  \newcommand{\xxxSubParagraphStar}[1]{\oldsubparagraph*{#1}\mbox{}}
  \newcommand{\xxxSubParagraphNoStar}[1]{\oldsubparagraph{#1}\mbox{}}
\fi
\makeatother


\usepackage{longtable,booktabs,array}
\usepackage{calc} % for calculating minipage widths
% Correct order of tables after \paragraph or \subparagraph
\usepackage{etoolbox}
\makeatletter
\patchcmd\longtable{\par}{\if@noskipsec\mbox{}\fi\par}{}{}
\makeatother
% Allow footnotes in longtable head/foot
\IfFileExists{footnotehyper.sty}{\usepackage{footnotehyper}}{\usepackage{footnote}}
\makesavenoteenv{longtable}
\usepackage{graphicx}
\makeatletter
\newsavebox\pandoc@box
\newcommand*\pandocbounded[1]{% scales image to fit in text height/width
  \sbox\pandoc@box{#1}%
  \Gscale@div\@tempa{\textheight}{\dimexpr\ht\pandoc@box+\dp\pandoc@box\relax}%
  \Gscale@div\@tempb{\linewidth}{\wd\pandoc@box}%
  \ifdim\@tempb\p@<\@tempa\p@\let\@tempa\@tempb\fi% select the smaller of both
  \ifdim\@tempa\p@<\p@\scalebox{\@tempa}{\usebox\pandoc@box}%
  \else\usebox{\pandoc@box}%
  \fi%
}
% Set default figure placement to htbp
\def\fps@figure{htbp}
\makeatother





\setlength{\emergencystretch}{3em} % prevent overfull lines

\providecommand{\tightlist}{%
  \setlength{\itemsep}{0pt}\setlength{\parskip}{0pt}}



 


\usepackage{eso-pic} % for background image
\usepackage{graphicx} % necessary for images
\usepackage{xifthen}
\usepackage{parskip} % for spacing
\usepackage{fancyhdr} % for fancy headers
\usepackage{lastpage} 
\usepackage{bera} % for font
\pagestyle{fancy}
\renewcommand{\headrulewidth}{0pt}
\fancyhead{} % footer for pages 2 on
\lfoot{} % Left footer
\cfoot{} % Change for center footer
\rfoot{\footnotesize Page \thepage\ of \pageref{LastPage}} % Right footer page #s 
\renewcommand{\footrulewidth}{0pt}
\makeatletter
\@ifpackageloaded{caption}{}{\usepackage{caption}}
\AtBeginDocument{%
\ifdefined\contentsname
  \renewcommand*\contentsname{Table of contents}
\else
  \newcommand\contentsname{Table of contents}
\fi
\ifdefined\listfigurename
  \renewcommand*\listfigurename{List of Figures}
\else
  \newcommand\listfigurename{List of Figures}
\fi
\ifdefined\listtablename
  \renewcommand*\listtablename{List of Tables}
\else
  \newcommand\listtablename{List of Tables}
\fi
\ifdefined\figurename
  \renewcommand*\figurename{Figure}
\else
  \newcommand\figurename{Figure}
\fi
\ifdefined\tablename
  \renewcommand*\tablename{Table}
\else
  \newcommand\tablename{Table}
\fi
}
\@ifpackageloaded{float}{}{\usepackage{float}}
\floatstyle{ruled}
\@ifundefined{c@chapter}{\newfloat{codelisting}{h}{lop}}{\newfloat{codelisting}{h}{lop}[chapter]}
\floatname{codelisting}{Listing}
\newcommand*\listoflistings{\listof{codelisting}{List of Listings}}
\makeatother
\makeatletter
\makeatother
\makeatletter
\@ifpackageloaded{caption}{}{\usepackage{caption}}
\@ifpackageloaded{subcaption}{}{\usepackage{subcaption}}
\makeatother
\usepackage{bookmark}
\IfFileExists{xurl.sty}{\usepackage{xurl}}{} % add URL line breaks if available
\urlstyle{same}
\hypersetup{
  pdftitle={Letter from ANU},
  pdfauthor={\hspace{0pt}},
  pdfsubject={Submission of Manuscript: Automated Assessment of Residual
Plots with Computer Vision Models},
  colorlinks=true,
  linkcolor={blue},
  filecolor={Maroon},
  citecolor={Blue},
  urlcolor={Blue},
  pdfcreator={LaTeX via pandoc}}


\title{Letter from ANU}
\author{\hspace{0pt}}
\date{July 25, 2025}
\begin{document}
\AddToShipoutPicture{\ifthenelse{\value{page}<2}{\includegraphics[width=\paperwidth,height=\paperheight]{\_extensions/anuopensci/quarto-anu-letter/assets/images/coverpage.jpg}}{\includegraphics[width=\paperwidth,height=\paperheight]{\_extensions/anuopensci/quarto-anu-letter/assets/images/page-background.jpg}}}


\signature{
\hspace{0pt}}
\address{Australian National University\\Research School of Finance,
Actuarial Studies and Statistics\\26C Kingsley
Street\\Canberra 2600\\Australia\\[2mm]}
\begin{letter}{Editors\\Journal of Computational and Graphical
Statistics\\ ~ \\Subject: Submission of Manuscript: Automated Assessment
of Residual Plots with Computer Vision Models}
\opening{Dear Prof.~Chen and Prof.~Sangalli,}


Please consider our manuscript ``Automated Assessment of Residual Plots
with Computer Vision Models'' for publication in the Journal of
Computational and Graphical Statistics.

This paper builds on our previous work, ``A Plot is Worth a Thousand
Tests: Assessing Residual Diagnostics with the Lineup Protocol'' (JCGS),
which provided empirical evidence for the effectiveness of visual
evaluation of residual plots through crowd-sourced lineup experiments.
While that study highlighted the cost and effort of human evaluations,
this work advances automation of residual plot diagnostics through an
application of a computer vision model. We specifically trained a model
to predict a custom distance measure that quantifies the discrepancy
between residuals and theoretically ``good'' distributions. Building on
these predictions, we construct a statistical testing framework aligned
with the original lineup protocol, ensuring a valid visual test. When
evaluated on the earlier experimenal data, the model performs slightly
worse than humans but surpasses conventional statistical tests due to
its reduced sensitivity to minor deviations from model assumptions. We
further demonstrate that the model, like human observers, relies on
visual patterns and shapes, making it a practical tool for reducing
manual effort in residual plot diagnostics. This approach would be of
particular interest to readers concerned with model diagnostics and the
automation of visual data analysis.

Thank you for the consideration of this manuscript. We believe that it
is a good fit for the Journal of Computational and Graphical Statistics.



\closing{Sincerely,}
\vfill
\end{letter}
\end{document}
