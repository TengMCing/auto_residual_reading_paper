% Options for packages loaded elsewhere
% Options for packages loaded elsewhere
\PassOptionsToPackage{unicode}{hyperref}
\PassOptionsToPackage{hyphens}{url}
\PassOptionsToPackage{dvipsnames,svgnames,x11names}{xcolor}
%
\documentclass[
  letterpaper,
  DIV=11,
  numbers=noendperiod]{scrartcl}
\usepackage{xcolor}
\usepackage{amsmath,amssymb}
\setcounter{secnumdepth}{-\maxdimen} % remove section numbering
\usepackage{iftex}
\ifPDFTeX
  \usepackage[T1]{fontenc}
  \usepackage[utf8]{inputenc}
  \usepackage{textcomp} % provide euro and other symbols
\else % if luatex or xetex
  \usepackage{unicode-math} % this also loads fontspec
  \defaultfontfeatures{Scale=MatchLowercase}
  \defaultfontfeatures[\rmfamily]{Ligatures=TeX,Scale=1}
\fi
\usepackage{lmodern}
\ifPDFTeX\else
  % xetex/luatex font selection
\fi
% Use upquote if available, for straight quotes in verbatim environments
\IfFileExists{upquote.sty}{\usepackage{upquote}}{}
\IfFileExists{microtype.sty}{% use microtype if available
  \usepackage[]{microtype}
  \UseMicrotypeSet[protrusion]{basicmath} % disable protrusion for tt fonts
}{}
\makeatletter
\@ifundefined{KOMAClassName}{% if non-KOMA class
  \IfFileExists{parskip.sty}{%
    \usepackage{parskip}
  }{% else
    \setlength{\parindent}{0pt}
    \setlength{\parskip}{6pt plus 2pt minus 1pt}}
}{% if KOMA class
  \KOMAoptions{parskip=half}}
\makeatother
% Make \paragraph and \subparagraph free-standing
\makeatletter
\ifx\paragraph\undefined\else
  \let\oldparagraph\paragraph
  \renewcommand{\paragraph}{
    \@ifstar
      \xxxParagraphStar
      \xxxParagraphNoStar
  }
  \newcommand{\xxxParagraphStar}[1]{\oldparagraph*{#1}\mbox{}}
  \newcommand{\xxxParagraphNoStar}[1]{\oldparagraph{#1}\mbox{}}
\fi
\ifx\subparagraph\undefined\else
  \let\oldsubparagraph\subparagraph
  \renewcommand{\subparagraph}{
    \@ifstar
      \xxxSubParagraphStar
      \xxxSubParagraphNoStar
  }
  \newcommand{\xxxSubParagraphStar}[1]{\oldsubparagraph*{#1}\mbox{}}
  \newcommand{\xxxSubParagraphNoStar}[1]{\oldsubparagraph{#1}\mbox{}}
\fi
\makeatother


\usepackage{longtable,booktabs,array}
\usepackage{calc} % for calculating minipage widths
% Correct order of tables after \paragraph or \subparagraph
\usepackage{etoolbox}
\makeatletter
\patchcmd\longtable{\par}{\if@noskipsec\mbox{}\fi\par}{}{}
\makeatother
% Allow footnotes in longtable head/foot
\IfFileExists{footnotehyper.sty}{\usepackage{footnotehyper}}{\usepackage{footnote}}
\makesavenoteenv{longtable}
\usepackage{graphicx}
\makeatletter
\newsavebox\pandoc@box
\newcommand*\pandocbounded[1]{% scales image to fit in text height/width
  \sbox\pandoc@box{#1}%
  \Gscale@div\@tempa{\textheight}{\dimexpr\ht\pandoc@box+\dp\pandoc@box\relax}%
  \Gscale@div\@tempb{\linewidth}{\wd\pandoc@box}%
  \ifdim\@tempb\p@<\@tempa\p@\let\@tempa\@tempb\fi% select the smaller of both
  \ifdim\@tempa\p@<\p@\scalebox{\@tempa}{\usebox\pandoc@box}%
  \else\usebox{\pandoc@box}%
  \fi%
}
% Set default figure placement to htbp
\def\fps@figure{htbp}
\makeatother





\setlength{\emergencystretch}{3em} % prevent overfull lines

\providecommand{\tightlist}{%
  \setlength{\itemsep}{0pt}\setlength{\parskip}{0pt}}



 


\usepackage[dvipsnames]{xcolor}
\newcommand{\review}[1]{{\color{violet} #1}}
\newenvironment{reviewer}{
\color{violet}}
{}
\renewenvironment{quote}{%
 \list{}{%
   \leftmargin0.5cm   % this is the adjusting screw
   \itshape
   \rightmargin\leftmargin
 }
 \item\relax
}
{\endlist}
\KOMAoption{captions}{tableheading}
\makeatletter
\@ifpackageloaded{caption}{}{\usepackage{caption}}
\AtBeginDocument{%
\ifdefined\contentsname
  \renewcommand*\contentsname{Table of contents}
\else
  \newcommand\contentsname{Table of contents}
\fi
\ifdefined\listfigurename
  \renewcommand*\listfigurename{List of Figures}
\else
  \newcommand\listfigurename{List of Figures}
\fi
\ifdefined\listtablename
  \renewcommand*\listtablename{List of Tables}
\else
  \newcommand\listtablename{List of Tables}
\fi
\ifdefined\figurename
  \renewcommand*\figurename{Figure}
\else
  \newcommand\figurename{Figure}
\fi
\ifdefined\tablename
  \renewcommand*\tablename{Table}
\else
  \newcommand\tablename{Table}
\fi
}
\@ifpackageloaded{float}{}{\usepackage{float}}
\floatstyle{ruled}
\@ifundefined{c@chapter}{\newfloat{codelisting}{h}{lop}}{\newfloat{codelisting}{h}{lop}[chapter]}
\floatname{codelisting}{Listing}
\newcommand*\listoflistings{\listof{codelisting}{List of Listings}}
\makeatother
\makeatletter
\makeatother
\makeatletter
\@ifpackageloaded{caption}{}{\usepackage{caption}}
\@ifpackageloaded{subcaption}{}{\usepackage{subcaption}}
\makeatother
\usepackage{bookmark}
\IfFileExists{xurl.sty}{\usepackage{xurl}}{} % add URL line breaks if available
\urlstyle{same}
\hypersetup{
  pdftitle={Automated Assessment of Residual Plots with Computer Vision Models},
  colorlinks=true,
  linkcolor={blue},
  filecolor={Maroon},
  citecolor={Blue},
  urlcolor={Blue},
  pdfcreator={LaTeX via pandoc}}


\title{Automated Assessment of Residual Plots with Computer Vision
Models}
\usepackage{etoolbox}
\makeatletter
\providecommand{\subtitle}[1]{% add subtitle to \maketitle
  \apptocmd{\@title}{\par {\large #1 \par}}{}{}
}
\makeatother
\subtitle{Response to reviewers}
\author{}
\date{2026-01-12}
\begin{document}
\maketitle


I thank the editor and the reviewer for their constructive comments that
have improved this paper. I have addressed all comments, colored in
\review{purple}. In addition, I also include an annotated version of the
paper (\texttt{diff.pdf}) with the difference between the original and
revised versions produced using \texttt{latexdiff}.

\subsection{Reviewer}\label{reviewer}

\subsubsection{General Comments}\label{general-comments}

\begin{reviewer}
The idea of the paper seems interesting: using computer vision to help
identify residual plots which indicates model violation. For any
computer vision task, the input is clearly the image. However, there
should be a clear definition of the outcome variable that the algorithm
wants to predict from images. From the current draft of this paper, it
is unclear that what is the outcome variable that the computer vision is
learning from the residual plots, how the training datasets are created
for computer vision learning, and how the performance of the computer
vision algorithm is assessed. Please see my detailed comments in items
10, 12,13. I suggest the authors make these basic setups explicitly
defined from the very beginning of the article.

\end{reviewer}

\subsubsection{Specific Comments}\label{specific-comments}

\begin{reviewer}

\begin{enumerate}
\def\labelenumi{\arabic{enumi}.}
\tightlist
\item
  p3, l9: What is ``the lineup protocol''?
\end{enumerate}

\end{reviewer}

\begin{reviewer}

\begin{enumerate}
\def\labelenumi{\arabic{enumi}.}
\setcounter{enumi}{1}
\tightlist
\item
  missing references: Loy and Hofmann (2013; 2014; 2015)
\end{enumerate}

\end{reviewer}

\begin{reviewer}

\begin{enumerate}
\def\labelenumi{\arabic{enumi}.}
\setcounter{enumi}{2}
\tightlist
\item
  p8 l16: Why do we need to replace it by a full-rank covariance matrix?
\end{enumerate}

\end{reviewer}

\begin{reviewer}

\begin{enumerate}
\def\labelenumi{\arabic{enumi}.}
\setcounter{enumi}{3}
\tightlist
\item
  eq (2), typically KL is specified by KL(p--- q) due to asymmetry
\end{enumerate}

\end{reviewer}

\begin{reviewer}

\begin{enumerate}
\def\labelenumi{\arabic{enumi}.}
\setcounter{enumi}{4}
\tightlist
\item
  p9 l 53: to solve eq 2: using ``evaluate'' for ``solve'' is better
\end{enumerate}

\end{reviewer}

\begin{reviewer}

\begin{enumerate}
\def\labelenumi{\arabic{enumi}.}
\setcounter{enumi}{5}
\tightlist
\item
  p19 l30: What is ``the data generating process''? We don't know the
  true distribution of y.
\end{enumerate}

\end{reviewer}

\begin{reviewer}

\begin{enumerate}
\def\labelenumi{\arabic{enumi}.}
\setcounter{enumi}{6}
\tightlist
\item
  p12, sec 5.1: this sounds like a standard simulation of the sampling
  distribution of ˆD in traditional statistics.
\end{enumerate}

\end{reviewer}

\begin{reviewer}

\begin{enumerate}
\def\labelenumi{\arabic{enumi}.}
\setcounter{enumi}{7}
\tightlist
\item
  p13, Sec 5.2: How do you do bootstrapping? We need to know the
  distribution of ˆD under the null. However, the given observed data y
  may not come from the null.
\end{enumerate}

\end{reviewer}

\begin{reviewer}

\begin{enumerate}
\def\labelenumi{\arabic{enumi}.}
\setcounter{enumi}{8}
\tightlist
\item
  Tbl1: it is unclear what this table is measuring. What is the R2
  measuring? What's the response and what is the predictors?
\end{enumerate}

\end{reviewer}

\begin{reviewer}

\begin{enumerate}
\def\labelenumi{\arabic{enumi}.}
\setcounter{enumi}{9}
\tightlist
\item
  Section 7 and 8: In computing ˆD, you need a P and Q for each targeted
  model violation. Is your computer vision learning algorithm targeted a
  particular model departure, eg, non-linearity or heteroskedasticity?
  What is your P and Q in generating the training data of ˆD and
  residual plots for computer vision learning? However, your results
  also show your performance for di!erent kinds of model violations.
  This is confusing. You need to specify clearly what is the ``true'' ˆD
  and what the ``predicted ˆD'' using computer vision in generating
  training data of ˆD and residual plots.
\end{enumerate}

\end{reviewer}

\begin{reviewer}

\begin{enumerate}
\def\labelenumi{\arabic{enumi}.}
\setcounter{enumi}{10}
\tightlist
\item
  There is no numbering for your equations.
\end{enumerate}

\end{reviewer}




\end{document}
