% interactcadsample.tex
% v1.04 - May 2023

\documentclass[]{interact}

\usepackage{epstopdf}% To incorporate .eps illustrations using PDFLaTeX, etc.
\usepackage{subfigure}% Support for small, `sub' figures and tables
%\usepackage[nolists,tablesfirst]{endfloat}% To `separate' figures and tables from text if required

\usepackage{natbib}% Citation support using natbib.sty
\bibpunct[, ]{(}{)}{;}{a}{}{,}% Citation support using natbib.sty
\renewcommand\bibfont{\fontsize{10}{12}\selectfont}% Bibliography support using natbib.sty

\theoremstyle{plain}% Theorem-like structures provided by amsthm.sty
\newtheorem{theorem}{Theorem}[section]
\newtheorem{lemma}[theorem]{Lemma}
\newtheorem{corollary}[theorem]{Corollary}
\newtheorem{proposition}[theorem]{Proposition}

\theoremstyle{definition}
\newtheorem{definition}[theorem]{Definition}
\newtheorem{example}[theorem]{Example}

\theoremstyle{remark}
\newtheorem{remark}{Remark}
\newtheorem{notation}{Notation}


% tightlist command for lists without linebreak
\providecommand{\tightlist}{%
  \setlength{\itemsep}{0pt}\setlength{\parskip}{0pt}}



\usepackage{lscape}
\usepackage{hyperref}
\usepackage[utf8]{inputenc}
\def\tightlist{}
\usepackage{setspace}
\doublespacing


\begin{document}


\articletype{}

\title{Appendix to ``Automated Assessment of Residual Plots with
Computer Vision Models''}


\author{\name{Weihao Li$^{a, b}$, Dianne Cook$^{a}$, Emi Tanaka$^{a, b,
c}$, Susan VanderPlas$^{d}$, Klaus Ackermann$^{a}$}
\affil{$^{a}$Department of Econometrics and Business Statistics, Monash
University, Clayton, VIC, Australia; $^{b}$Biological Data Science
Institute, Australian National University, Acton, ACT,
Australia; $^{c}$Research School of Finance, Actuarial Studies and
Statistics, Australian National University, Acton, ACT,
Australia; $^{d}$Department of Statistics, University of Nebraska,
Lincoln, Nebraska, USA}
}


\maketitle

\begin{abstract}
Plotting the residuals is a recommended procedure to diagnose deviations
from linear model assumptions, such as non-linearity,
heteroscedasticity, and non-normality. The presence of structure in
residual plots can be tested using the lineup protocol to do visual
inference. There are a variety of conventional residual tests, but the
lineup protocol, used as a statistical test, performs better for
diagnostic purposes because it is less sensitive and applies more
broadly to different types of departures. However, the lineup protocol
relies on human judgment which limits its scalability. This work
presents a solution by providing a computer vision model to automate the
assessment of residual plots. It is trained to predict a distance
measure that quantifies the disparity between the residual distribution
of a fitted classical normal linear regression model and the reference
distribution, based on Kullback-Leibler divergence. From extensive
simulation studies, the computer vision model exhibits lower sensitivity
than conventional tests but higher sensitivity than human visual tests.
It is slightly less effective on non-linearity patterns. Several
examples from classical papers and contemporary data illustrate the new
procedures, highlighting its usefulness in automating the diagnostic
process and supplementing existing methods.
\end{abstract}

\begin{keywords}
statistical graphics; data visualization; visual inference; computer
vision; machine learning; hypothesis testing; reression analysis;
cognitive perception; simulation; practical significance
\end{keywords}



\bibliographystyle{tfcad}
\bibliography{bibliography.bib}





\end{document}
