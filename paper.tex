% interactcadsample.tex
% v1.04 - May 2023

\documentclass[]{interact}

\usepackage{epstopdf}% To incorporate .eps illustrations using PDFLaTeX, etc.
\usepackage{subfigure}% Support for small, `sub' figures and tables
%\usepackage[nolists,tablesfirst]{endfloat}% To `separate' figures and tables from text if required

\usepackage{natbib}% Citation support using natbib.sty
\bibpunct[, ]{(}{)}{;}{a}{}{,}% Citation support using natbib.sty
\renewcommand\bibfont{\fontsize{10}{12}\selectfont}% Bibliography support using natbib.sty

\theoremstyle{plain}% Theorem-like structures provided by amsthm.sty
\newtheorem{theorem}{Theorem}[section]
\newtheorem{lemma}[theorem]{Lemma}
\newtheorem{corollary}[theorem]{Corollary}
\newtheorem{proposition}[theorem]{Proposition}

\theoremstyle{definition}
\newtheorem{definition}[theorem]{Definition}
\newtheorem{example}[theorem]{Example}

\theoremstyle{remark}
\newtheorem{remark}{Remark}
\newtheorem{notation}{Notation}


% tightlist command for lists without linebreak
\providecommand{\tightlist}{%
  \setlength{\itemsep}{0pt}\setlength{\parskip}{0pt}}



\usepackage{lscape}
\usepackage{hyperref}
\usepackage[utf8]{inputenc}
\def\tightlist{}
\usepackage{setspace}
\doublespacing

\usepackage{booktabs}
\usepackage{longtable}
\usepackage{array}
\usepackage{multirow}
\usepackage{wrapfig}
\usepackage{float}
\usepackage{colortbl}
\usepackage{pdflscape}
\usepackage{tabu}
\usepackage{threeparttable}
\usepackage{threeparttablex}
\usepackage[normalem]{ulem}
\usepackage{makecell}
\usepackage{xcolor}

\begin{document}


\articletype{}

\title{Automated Assessment of Residual Plots with Computer Vision
Models}


\author{\name{Weihao Li$^{a, b}$, Dianne Cook$^{a}$, Emi Tanaka$^{a, b,
c}$, Susan VanderPlas$^{d}$, Klaus Ackermann$^{a}$}
\affil{$^{a}$Department of Econometrics and Business Statistics, Monash
University, Clayton, VIC, Australia; $^{b}$Biological Data Science
Institute, Australian National University, Acton, ACT,
Australia; $^{c}$Research School of Finance, Actuarial Studies and
Statistics, Australian National University, Acton, ACT,
Australia; $^{d}$Department of Statistics, University of Nebraska,
Lincoln, Nebraska, USA}
}


\maketitle

\begin{abstract}
Plotting the residuals is a recommended procedure to diagnose deviations
from linear model assumptions, such as non-linearity,
heteroscedasticity, and non-normality. The presence of structure in
residual plots can be tested using the lineup protocol to do visual
inference. There are a variety of conventional residual tests, but the
lineup protocol, used as a statistical test, performs better for
diagnostic purposes because it is less sensitive and applies more
broadly to different types of departures. However, the lineup protocol
relies on human judgment which limits its scalability. This work
presents a solution by providing a computer vision model to automate the
assessment of residual plots. It is trained to predict a distance
measure that quantifies the disparity between the residual distribution
of a fitted classical normal linear regression model and the reference
distribution, based on Kullback-Leibler divergence. From extensive
simulation studies, the computer vision model exhibits lower sensitivity
than conventional tests but higher sensitivity than human visual tests.
It is slightly less effective on non-linearity patterns. Several
examples from classical papers and contemporary data illustrate the new
procedures, highlighting its usefulness in automating the diagnostic
process and supplementing existing methods.
\end{abstract}

\begin{keywords}
statistical graphics; data visualization; visual inference; computer
vision; machine learning; hypothesis testing; reression analysis;
cognitive perception; simulation; practical significance
\end{keywords}

\section{Introduction}\label{sec-model-introduction}

Plotting residuals is commonly regarded as a standard practice in linear
regression diagnostics \citep{belsley1980regression, cook1982residuals}.
This visual assessment plays a crucial role in identifying whether model
assumptions, such as linearity, homoscedasticity, and normality, are
reasonable. It also helps in understanding the goodness of fit and
various unexpected characteristics of the model.

Generating a residual plot in most statistical software is often as
straightforward as executing a line of code or clicking a button.
However, accurately interpreting a residual plot can be challenging. A
residual plot can exhibit various visual features, but it is crucial to
recognize that some may arise from the characteristics of predictors and
the natural stochastic variation of the observational unit, rather than
indicating a violation of model assumptions \citep{li2024plot}. Consider
Figure \ref{fig:false-finding} as an example, the residual plot displays
a triangular left-pointing shape. The distinct difference in the spread
of the residuals across the fitted values may result in the analyst
suggesting that there may be heteroskedasticity, however, it is
important to avoid over-interpreting this visual pattern. In this case,
the fitted regression model is correctly specified, and the triangular
shape is actually a result of the skewed distribution of the predictors,
rather than indicating a flaw in the model.

The concept of visual inference, as proposed by
\citet{buja2009statistical}, provides an inferential framework to assess
whether residual plots indeed contain visual patterns inconsistent with
the model assumptions. The fundamental idea involves testing whether the
true residual plot visually differs significantly from null plots, where
null plots are plotted with residuals generated from the residual
rotation distribution \citep{langsrud2005rotation}, which is a
distribution consistent with the null hypothesis \(H_0\) that the linear
regression model is correctly specified. Typically, the visual test is
accomplished through the lineup protocol, where the true residual plot
is embedded within a lineup alongside several null plots. If the true
residual plot can be distinguished from the lineup, it provides evidence
for rejecting \(H_0\).

The practice of delivering a residual plot as a lineup is generally
regarded as a valuable approach. Beyond its application in residual
diagnostics, the lineup protocol has been integrated into the analysis
of diverse subjects. For instance, Loy and Hofmann
\citetext{\citeyear{loy2013diagnostic}; \citeyear{loy2014hlmdiag}; \citeyear{loy2015you}}
illustrated its applicability in diagnosing hierarchical linear models.
Additionally, \citet{widen2016graphical} and \citet{fieberg2024using}
demonstrated its utility in geographical and ecology research
respectively, while \citet{krishnan2021hierarchical} explored its
effectiveness in forensic examinations.

A practical limitation of the lineup protocol lies in its reliance on
human judgements \citep[see][ about the practical
limitations]{li2024plot}. Unlike conventional statistical tests that can
be performed computationally in statistical software, the lineup
protocol requires human evaluation of images. This characteristic makes
it less suitable for large-scale applications, given the associated high
labour costs and time requirements. There is a substantial need to
develop an approach to substitute these human judgement with an
automated reading of data plots using machines.

The utilization of computers to interpret data plots has a rich history,
with early efforts such as ``Scagnostics'' by \citet{tukey1985computer},
a set of numerical statistics that summarize features of scatter plots.
\citet{wilkinson2005graph} expanded on this work, introducing
scagnostics based on computable measures applied to planar proximity
graphs. These measures, including, but not limited to, ``Outlying'',
``Skinny'', ``Stringy'', ``Straight'', ``Monotonic'', ``Skewed'',
``Clumpy'', and ``Striated'', aimed to characterize outliers, shape,
density, trend, coherence and other characteristics of the data. While
this approach has been inspiring, there is a recognition
\citep{buja2009statistical} that it may not capture all the necessary
visual features that differentiate true residual plots from null plots.
A more promising alternative entails enabling machines to learn the
function for extracting visual features from residual plots.
Essentially, this means empowering computers to discern the crucial
visual features for residual diagnostics and determining the method to
extract them.

Modern computer vision models are well-suited for addressing this
challenge. They rely on deep neural networks with convolutional layers
\citep{fukushima1982neocognitron}. These layers use small, sliding
windows to scan the image, performing a dot product to extract local
features and patterns. Numerous studies have demonstrated the efficacy
of convolutional layers in addressing various vision tasks, including
image recognition \citep{rawat2017deep}. Despite the widespread use of
computer vision models in fields like computer-aided diagnosis
\citep{lee2015image}, pedestrian detection \citep{brunetti2018computer},
and facial recognition \citep{emami2012facial}, their application in
reading data plots remains limited. While some studies have explored the
use of computer vision models for tasks such as reading recurrence plots
for time series regression \citep{ojeda2020multivariate}, time series
classification
\citep{chu2019automatic, hailesilassie2019financial, hatami2018classification, zhang2020encoding},
anomaly detection \citep{chen2020convolutional}, and pairwise causality
analysis \citep{singh2017deep}, the application of reading residual
plots with computer vision models is a new field of study.

In this paper, we develop computer vision models and integrate them into
the residual plots diagnostics workflow, addressing the need for an
automated visual inference. The paper is structured as follows. Section
\ref{sec-model-specifications} discusses various specifications of the
computer vision models. Section
\ref{sec-model-distance-between-residual-plots} defines the distance
measure used to detect model violations, while Section
\ref{sec-model-distance-estimation} explains how the computer vision
models estimate this distance measure. Section
\ref{sec-model-statistical-testing} covers the statistical tests based
on the estimated distance, and Section \ref{sec-model-violations-index}
introduces a Model Violations Index, which offers a quicker and more
convenient assessment. Sections \ref{sec-model-architecture} and
\ref{sec-model-data-generation} describe the model architecture and the
corresponding data generation and training process, respectively. The
results are presented in Section \ref{sec-model-results}. Example
dataset applications are discussed in Section \ref{sec-examples}.
Finally, we conclude with a discussion of our findings and propose ideas
for future research directions.

\begin{figure}[!h]

{\centering \includegraphics[width=1\linewidth]{paper_files/figure-latex/false-finding-1} 

}

\caption{An example residual vs fitted values plot (red line indicates 0 corresponds to the x-intercept, i.e. $y=0$). The vertical spread of the data points varies with the fitted values. This often indicates the existence of heteroskedasticity, however, here the result is due to skewed distribution of the predictors rather than heteroskedasticity. The Breusch-Pagan test rejects this residual plot at 95\% significance level ($p\text{-value} = 0.046$).}\label{fig:false-finding}
\end{figure}

\section{Model Specifications}\label{sec-model-specifications}

There are various specifications of the computer vision model that can
be used to assess residual plots. We discuss these specifications below
focusing on two key components of the model formula: the input and the
output format.

\subsection{Input Formats}\label{input-formats}

Deep learning models are sensitive to input design, and several
alternatives are available for encoding residual plots.

A simple approach involves feeding vectors of residuals and fitted
values into the model. While this format contains all relevant
information, the input length varies with sample size, which conflicts
with the fixed-shape requirements of modern vision models such as those
implemented in TensorFlow \citep{abadi2016tensorflow}. Padding can be
used to enforce uniform length but risks truncation if inputs exceed the
fixed size. Alternatively, fixed-size sampling of residual--fitted value
pairs can ensure consistency, though at the cost of some information
loss.

Another strategy is to convert residual plots into images. This enables
the use of standard convolutional architectures such as VGG16
\citep{simonyan2014very}, which can effectively capture spatial
patterns. Although discretization introduces some losses in detail,
image-based inputs offer a richer representation by preserving the joint
distribution of residuals and fitted values in a visually interpretable
format. To ensure model generalization, it is essential to use residual
plots in a consistent style, i.e., same aesthetics, layout, scaling and
color schemes.

Using multiple residual plots as input, such as pairs, triplets, or full
lineups is also a possible option. For instance, triplet networks
\citep{chopra2005learning} can assess visual similarity if we assume
null plots are more similar from each other than from the true residual
plot, which may help distinguish null plots from true residual plots.
However, these architectures introduce complexity in training due to
weight sharing and specialized loss functions. We experimented with this
setting in a pilot study, and found multiple residual plots led to high
input resolution, high computational cost and suboptimal model
performance.

Balancing accuracy, interpretability, and implementation cost, we adopt
a single residual plot in image format as the model input.

\subsection{Output Formats}\label{output-formats}

Given the input is a single fixed-resolution image, the model output can
be defined for various tasks, including binary classification,
multiclass classification, and numeric regression.

In the binary case, the outcome may indicate whether the input image is
consistent with a null plot, as determined either by (1) the
data-generating process or (2) the outcome of a visual test based on
human judgment. The first approach models the probability that a
residual plot is not a null plot and is a viable option. In contrast,
the second relies on experimental data that are often scarce or
inconsistent across studies.

Alternatively, a multiclass outcome may classify images into categories
such as ``contains outliers'' vs.~``does not contain outliers'' or ``is
non-linear'' vs.~``not non-linear,'' among other diagnostic features.
While this approach is appealing in theory, it essentially reverts to
the traditional testing framework, where tests focus on detecting
specific types of model violations in isolation.

A third option is to produce a meaninful and interpretable numerical
measure that quantifies the strength of suspicious visual patterns
reflecting the extent of model violations, or the difficulty index for
identifying whether a residual plot has no issues. However, such
measures are often used informally in daily communications but rarely
formalized. For supervised learning, they must be clearly defined and
reliably estimable.

In this study, we defined and used a continuous distance between a true
residual plot and a theoretically ``good'' residual plot. This approach
captures the degree to which a plot deviates from model assumptions. In
contrast, a probability output based on binary labels, where true
residual plots are labeled as 1 and null plots as 0, cannot express the
severity of violation. It forces the model to infer this information
solely from the data distribution, which may be inefficient and discards
useful prior knowledge.

\section{Distance from a Theoretically ``Good'' Residual
Plot}\label{sec-model-distance-between-residual-plots}

To develop a computer vision model for assessing residual plots within
the visual inference framework, it is important to precisely define a
numerical measure of ``difference'' or ``distance'' between plots. This
distance can take the form of a basic statistical operation on pixels,
such as the sum of square differences, however, a pixel-to-pixel
comparison makes little sense in comparing residual plots where the main
interest would be structural patterns. Alternatively, it could involve
established image similarity metrics like the Structural Similarity
Index Measure \citep{wang2004image} which compares images by integrating
three perception features of an image: contrast, luminance, and
structure (related to average, standard deviation and correlation of
pixel values over a window, respectively). These image similarity
metrics are tailored for image comparison in vastly different tasks to
evaluating data plots, where only essential plot elements require
assessment \citep{chowdhury2018measuring}. We can alternatively define a
notion of distance by integrating key plot elements (instead of key
perception features like luminance, contrast, and structure), such as
those captured by scagnostics mentioned in Section
\ref{sec-model-introduction}, but the functional form still needs to be
carefully refined to accurately reflect the extent of the violations.

In this section, we introduce a distance measure between a true residual
plot and a theoretically `good' residual plot. This measure quantifies
the divergence between the residual distribution of a given fitted
regression model and that of a correctly specified model. The
computation assumes knowledge of the data generating processes for
predictors and response variables. Since these processes are often
unknown in practice, we will discuss a method to estimate this distance
using a computer vision model in Section
\ref{sec-model-distance-estimation}.

\subsection{Residual Distribution}\label{residual-distribution}

For a classical normal linear regression model,
\(\boldsymbol{y} = \boldsymbol{X}\boldsymbol{\beta} + \boldsymbol{e}\),
the residual \(\hat{\boldsymbol{e}}\) are derived as the difference of
the fitted values and observed values \(\boldsymbol{y}\). Suppose the
data generating process is known and the regression model is correctly
specified, by the Frisch-Waugh-Lowell theorem \citep{frisch1933partial},
residuals \(\hat{\boldsymbol{e}}\) can also be treated as random
variables and written as a linear transformation of the error
\(\boldsymbol{e}\) formulated as
\(\hat{\boldsymbol{e}} = \boldsymbol{R}\boldsymbol{e}\), where
\(\boldsymbol{R}=\boldsymbol{I}_n -\boldsymbol{X}(\boldsymbol{X}^\top\boldsymbol{X})^{-1}\boldsymbol{X}^\top\)
is the residual operator, \(\boldsymbol{I}_n\) is a \(n\) by \(n\)
identity matrix, and \(n\) is the number of observations.

One of the assumptions of the classical normal linear regression model
is that the error \(\boldsymbol{e}\) follows a multivariate normal
distribution with zero mean and constant variance, i.e.,
\(\boldsymbol{e} \sim N(\boldsymbol{0}_n,\sigma^2\boldsymbol{I}_n)\). It
follows that the distribution of residuals \(\hat{\boldsymbol{e}}\) can
be characterized by a certain probability distribution, denoted as
\(Q\), which is transformed from the multivariate normal distribution.
This reference distribution \(Q\) summarizes what ``good'' residuals
should follow given the design matrix \(\boldsymbol{X}\) is known and
fixed.

Suppose the design matrix \(\boldsymbol{X}\) has linearly independent
columns, the trace of the hat matrix
\(\boldsymbol{H} = \boldsymbol{X}(\boldsymbol{X}^\top\boldsymbol{X})^{-1}\boldsymbol{X}^\top\)
will equal to the number of columns in \(\boldsymbol{X}\) denoted as
\(k\). As a result, the rank of \(\boldsymbol{R}\) is \(n - k\), and
\(Q\) is a degenerate multivariate distribution. To capture the
characteristics of \(Q\), such as moments, we can simulate a large
numbers of \(\boldsymbol{\varepsilon}\) and transform it to
\(\boldsymbol{e}\) to get the empirical estimates. For simplicity, in
this study, we replaced the variance-covariance matrix of residuals
\(\text{cov}(\boldsymbol{e}, \boldsymbol{e}) = \boldsymbol{R}\sigma^2\boldsymbol{R}^\top = \boldsymbol{R}\sigma^2\)
with a full-rank diagonal matrix
\(\text{diag}(\boldsymbol{R}\sigma^2)\), where \(\text{diag}(.)\) sets
the non-diagonal entries of a matrix to zeros. The resulting
distribution for \(Q\) is
\(N(\boldsymbol{0}_n, \text{diag}(\boldsymbol{R}\sigma^2))\).

Distribution \(Q\) is derived from the correctly specified model.
However, if the model is misspecified, then the actual distribution of
residuals denoted as \(P\), will be different from \(Q\). For example,
if the data generating process contains variables correlated with any
column of \(\boldsymbol{X}\) but missing from \(\boldsymbol{X}\),
causing an omitted variable problem, \(P\) will be different from \(Q\)
because the residual operator obtained from the fitted regression model
will not be the same as \(\boldsymbol{R}\). Besides, if the
\(\boldsymbol{\varepsilon}\) follows a non-normal distribution such as a
multivariate lognormal distribution, \(P\) will usually be skewed and
has a long tail.

\subsection{\texorpdfstring{Distance of \(P\) from
\(Q\)}{Distance of P from Q}}\label{distance-of-p-from-q}

Defining a proper distance between distributions is usually easier than
defining a proper distance between data plots. Given the true residual
distribution \(Q\) and the reference residual distribution \(P\), we
used a distance measure based on Kullback-Leibler divergence
\citep{kullback1951information} to quantify the difference between two
distributions as

\begin{equation} \label{eq:kl-0}
D = \log\left(1 + D_{KL}\right),
\end{equation}

where \(D_{KL}\) is defined as

\begin{equation} \label{eq:kl-1}
D_{KL} = \int_{\mathbb{R}^{n}}\log\frac{p(\boldsymbol{e})}{q(\boldsymbol{e})}p(\boldsymbol{e})d\boldsymbol{e},
\end{equation}

\noindent and \(p(.)\) and \(q(.)\) are the probability density
functions for distribution \(P\) and distribution \(Q\), respectively.

This distance measure was first proposed in \citet{li2024plot}. It was
mainly designed for measuring the effect size of non-linearity and
heteroskedasticity in a residual plot. \citet{li2024plot} have derived
that, for a classical normal linear regression model that omits
necessary higher-order predictors \(\boldsymbol{Z}\) and the
corresponding parameter \(\boldsymbol{\beta}_z\), and incorrectly
assumes
\(\boldsymbol{\varepsilon} \sim N(\boldsymbol{0}_n,\sigma^2\boldsymbol{I}_n)\)
while in fact
\(\boldsymbol{\varepsilon} \sim N(\boldsymbol{0}_n, \boldsymbol{V})\)
where \(\boldsymbol{V}\) is an arbitrary symmetric positive
semi-definite matrix, \(Q\) can be represented as
\(N(\boldsymbol{R}\boldsymbol{Z}\boldsymbol{\beta}_z, \text{diag}(\boldsymbol{R}\boldsymbol{V}\boldsymbol{R}))\).
Note that the variance-covariance matrix is replaced with the diagonal
matrix to ensure it is a full-rank matrix.

Since both \(P\) and \(Q\) are adjusted to be multivariate normal
distributions, Equation \ref{eq:kl-1} can be further expanded to

\begin{equation} \label{eq:kl-2}
D_{KL} = \frac{1}{2}\left(\log\frac{|\boldsymbol{W}|}{|\text{diag}(\boldsymbol{R}\sigma^2)|} - n + \text{tr}(\boldsymbol{W}^{-1}\text{diag}(\boldsymbol{R}\sigma^2)) + \boldsymbol{\mu}_z^\top\boldsymbol{W}^{-1}\boldsymbol{\mu}_z\right),
\end{equation}

\noindent where
\(\boldsymbol{\mu}_z = \boldsymbol{R}\boldsymbol{Z}\boldsymbol{\beta}_z\),
and
\(\boldsymbol{W} = \text{diag}(\boldsymbol{R}\boldsymbol{V}\boldsymbol{R})\).
The assumed error variance \(\sigma^2\) is set to be
\(\text{tr}(\boldsymbol{V})/n\), which is the expectation of the
estimated variance.

\subsection{\texorpdfstring{Non-normal
\(P\)}{Non-normal P}}\label{non-normal-p}

For non-normal error \(\boldsymbol{\varepsilon}\), the true residual
distribution \(P\) is unlikely to be a multivariate normal distribution.
Thus, Equation \ref{eq:kl-2} given in \citet{li2024plot} will not be
applicable to models violating the normality assumption.

To evaluate the Kullback-Leibler divergence of non-normal \(P\) from
\(Q\), the fallback is to solve Equation \ref{eq:kl-1} numerically.
However, since \(\boldsymbol{e}\) is a linear transformation of
non-normal random variables, it is very common that the general form of
\(P\) is unknown, meaning that we can not easily compute
\(p(\boldsymbol{e})\) using a well-known probability density function.
Additionally, even if \(p(\boldsymbol{e})\) can be calculated for any
\(\boldsymbol{e} \in \mathbb{R}^n\), it will be very difficult to do
numerical integration over the \(n\)-dimensional space, because \(n\)
could be potentially very large.

In order to approximate \(D_{KL}\) in a practically computable manner,
the elements of \(\boldsymbol{e}\) are assumed to be independent of each
other. This assumption solves both of the issues mentioned above. First,
we no longer need to integrate over \(n\) random variables. The result
of Equation \ref{eq:kl-1} is now the sum of the Kullback-Leibler
divergence evaluated for each individual residual due to the assumption
of independence between observations. Second, it is not required to know
the joint probability density \(p(\boldsymbol{e})\) any more. Instead,
the evaluation of Kullback-Leibler divergence for an individual residual
relies on the knowledge of the marginal density \(p_i(e_i)\), where
\(e_i\) is the \(i\)-th residual for \(i = 1, ..., n\). This is much
easier to approximate through simulation. It is also worth mentioning
that this independence assumption generally will not hold if
\(\text{cov}(e_i, e_j) \neq 0\) for any \(1 \leq i < j \leq n\), but its
existence is essential for reducing the computational cost.

Given \(\boldsymbol{X}\) and \(\boldsymbol{\beta}\), the algorithm for
approximating Equation \ref{eq:kl-1} starts from simulating \(m\) sets
of observed values \(\boldsymbol{y}\) according to the data generating
process. The observed values are stored in a matrix \(\boldsymbol{A}\)
with \(n\) rows and \(m\) columns, where each column of
\(\boldsymbol{A}\) is a set of observed values. Then, we can get \(m\)
sets of realized values of \(\boldsymbol{e}\) stored in the matrix
\(\boldsymbol{B}\) by applying the residual operator
\(\boldsymbol{B} = \boldsymbol{R}\boldsymbol{A}\). Furthermore, kernel
density estimation (KDE) with Gaussian kernel and optimal bandwidth
selected by the Silverman's rule of thumb \citep{silverman2018density}
is applied on each row of \(\boldsymbol{B}\) to estimate \(p_i(e_i)\)
for \(i = 1, ..., n\). The KDE computation can be done by the
\texttt{density} function in R.

Since the Kullback-Leibler divergence can be viewed as the expectation
of the log-likelihood ratio between distribution \(P\) and distribution
\(Q\) evaluated on distribution \(P\), we can reuse the simulated
residuals in matrix \(\boldsymbol{B}\) to estimate the expectation by
the sample mean. With the independence assumption, for non-normal \(P\),
\(D_{KL}\) can be approximated by

\begin{align*} \label{eq:kl-3}
D_{KL} &\approx \sum_{i = 1}^{n} \hat{D}_{KL}^{(i)}, \\
\hat{D}_{KL}^{(i)} &= \frac{1}{m}\sum_{j = 1}^{m} \log\frac{\hat{p}_i(B_{ij})}{q(B_{ij})},
\end{align*}

\noindent where \(\hat{D}_{KL}^{(i)}\) is the estimator of the
Kullback-Leibler divergence for an individual residual \(e_i\),
\(\boldsymbol{B}_{ij}\) is the \(i\)-th row and \(j\)-th column entry of
the matrix \(\boldsymbol{B}\), \(\hat{p}_i(.)\) is the kernel density
estimator of \(p_i(.)\), \(q(.)\) is the normal density function with
mean zero and an assumed variance estimated as
\(\hat{\sigma}^2 = \sum_{b \in vec(\boldsymbol{B})}(b - \sum_{b \in vec(\boldsymbol{B})} b/nm)^2/(nm - 1)\),
and \(vec(.)\) is the vectorization operator which turns a
\(n \times m\) matrix into a \(nm \times 1\) column vector by stacking
the columns of the matrix on top of each other.

\section{Distance Estimation}\label{sec-model-distance-estimation}

We previously defined a distance measure (Equation \ref{eq:kl-0}) to
quantify the difference between the true residual distribution \(P\) and
an ideal reference distribution \(Q\). However, this distance measure
can only be computed when the data generating process is known. In
reality, we often have no knowledge about the data generating process,
otherwise we do not need to do a residual diagnostic in the first place.

To approximate this distance from a residual plot, we proposed a
computer vision estimator \(\hat{D}\) formulated as

\begin{equation} \label{eq:d-approx}
\hat{D} = f_{CV}(V_{h \times w}(\boldsymbol{e}, \hat{\boldsymbol{y}})),
\end{equation}

\noindent where \(V_{h \times w}(.)\) renders the residuals vs fitted
values plot as an \(h \times w\) RGB image, and \(f_{CV}(.)\) is a model
that maps the image to an estimated distance
\(\hat{D} \in [0, +\infty)\).

The distance estimator \(\hat{D}\) allows us to assess how closely the
residuals resemble an ideal distribution and use as an index of model
violation severity. However, \(\hat{D}\) is not expected to equal the
true distance \(D\), as a single residual plot may not capture all
characteristics of the residual distribution. For the same distribution
\(P\), simulated plots can vary visually, especially with small \(n\),
leading to estimation error that depends on how representative the input
plot is.

\section{Statistical Testing}\label{sec-model-statistical-testing}

\subsection{Lineup Evaluation}\label{sec-model-lineup-evaluation}

Theoretically, the distance \(D\) for a correctly specified model is
\(0\), as \(P = Q\). However, a computer vision model may not predict
\(\hat{D} = 0\) for a null plot. For example, Figure
\ref{fig:false-finding} shows a null plot with a pattern suggestive of
heteroskedasticity. The model can not discern whether such patterns
arise from heteroskedasticity or from skewed fitted values.
Additionally, some null plots could have outliers or strong visual
patterns due to randomness, and a reasonable model will try to summarize
those information into the prediction, resulting in \(\hat{D} > 0\).
This is not problematic when \(\hat{D}\) is large, as strong visual
signals typically justify rejection of \(H_0\). But when \(\hat{D}\) is
moderate, it is insufficient alone for decision-making.

To address this issue we can adhere to the paradigm of visual inference,
by comparing the estimated distance \(\hat{D}\) to the estimated
distances for the null plots in a lineup. Specifically, if a lineup
comprises 20 plots, \(H_0\) will be rejected if \(\hat{D}\) exceeds the
maximum estimated distance among the \(m - 1\) null plots, denoted as
\(\max\limits_{1 \leq i \leq m-1} {\hat{D}_{null}^{(i)}}\), where
\(\hat{D}_{null}^{(i)}\) represents the estimated distance for the
\(i\)-th null plot. This approach is conceptually equivalent to the
typical lineup protocol requiring a 95\% significance level, where
\(H_0\) is rejected if the data plot is identified as the most distinct
plot by the sole observer.

Moreover, if the number of plots in a lineup, denoted by \(m\), is
sufficiently large, the empirical distribution of
\({\hat{D}_{null}^{(i)}}\) can be viewed as an approximation of the null
distribution of the estimated distance. Consequently, quantiles of the
null distribution can be estimated using the sample quantiles and used
for decision-making purposes. The details of the sample quantile
computation can be found in \citet{hyndman1996sample}. For instance, if
\(\hat{D}\) is greater than or equal to the 95\% sample quantile,
denoted as \(Q_{null}(0.95)\), we can conclude that the estimated
distance for the true residual plot is significantly different from the
estimated distance for null plots with a 95\% significance level. Based
on our experience, at least 100 null plots are typically needed for
stable estimation, but more may be required if the null distribution has
heavy tails. Alternatively, a \(p\)-value is the probability of
observing a distance equally or greater than \(\hat{D}\) under \(H_0\),
and it can be estimated as
\(\frac{1}{m} + \frac{1}{m}\sum_{i=1}^{m-1}I\left(\hat{D}_{null}^{(i)} \geq \hat{D}\right)\).

To reduce computational cost, a pre-computed lattice of \(\hat{D}\)
quantiles under \(H_0\) for various sample sizes can be used. By
matching observed values to the closest entries in this lattice,
approximate \(p\)-values can be obtained efficiently, with minor loss of
precision.

\subsection{Bootstrapping}\label{bootstrapping}

Bootstrap methods are commonly used in linear regression to estimate
parameter variability without strong distributional assumptions
\citep{davison1997bootstrap, efron1994introduction}. This involves
resampling observations with replacement and refitting the model.

Similarly, we can apply bootstrapping to the estimated distance
\(\hat{D}\). For each bootstrap sample \(i = 1, \dots, n_{boot}\), we
obtain a refitted model \(M^{(i)}_{boot}\), a corresponding residual
plot \(V^{(i)}_{boot}\), and a predicted distance
\(\hat{D}^{(i)}_{boot}\). If we are interested in the variation of
\(\hat{D}\), the distribution of \(\hat{D}^{(i)}_{boot}\) can be used to
construct confidence intervals.

Alternatively, since each \(M_{boot}^{(i)}\) has its own residual
distribution, a new approximated null distribution can be construed and
the corresponding 95\% sample quantile \(Q_{boot}^{(i)}(0.95)\) can be
computed. Then, if \(\hat{D}_{boot}^{(i)} \geq Q_{boot}^{(i)}(0.95)\),
\(H_0\) will be rejected for \(M_{boot}^{(i)}\). The ratio of rejected
\(M_{boot}^{(i)}\) among all the refitted models reflects how often the
assumed regression model are considered to be misspecified if data were
repeatedly drawn from the same process. But this approach is
computationally very expensive because it requires
\(n_{boot} \times (n_{null} + 1)\) times of residual plot assessment. In
practice, \(Q_{null}(0.95)\) can be used to replace
\(Q_{boot}^{(i)}(0.95)\) in the computation.

\section{Model Violations Index}\label{sec-model-violations-index}

In \ref{sec-model-lineup-evaluation}, we noted that a pre-computed
lattice of \(\hat{D}\) quantiles can reduce the computation cost of
lineup tests. Another practical approach is to assess model performance
directly using the value of \(\hat{D}\).

The estimator \(\hat{D}\) captures the difference between the true and
reference residual distributions, which reflects the extent of model
violations, making it instrumental in forming a model violations index
(MVI). However, when more observations are used in regression, the value
of \(\hat{D}\) tends to increase logarithmically. This is because
\(D = \log(1 + D_{KL})\), and under the assumption of independence,
\(D_{KL}\) is the sum of \(D_{KL}^{(i)}\) across all observations. This
does not mean that \(\hat{D}\) becomes less reliable. In fact, larger
samples often make model violations more visible in residual plots,
unless strong overlapping masks the patterns.

However, to create a standardized and generalizable index, it is
important to adjust for the effect of sample size. Therefore, the Model
Violations Index (MVI) is proposed as

\begin{equation} \label{eq:mvi}
\text{MVI} = C + \hat{D} - \log(n),
\end{equation}

\noindent where \(C\) is a sufficiently large constant to ensure the
result remains positive, and the \(\log(n)\) term offset the increase in
\(D\) with larger sample sizes.

Figure \ref{fig:poly-heter-index} displays the residual plots for fitted
models exhibiting varying degrees of non-linearity and
heteroskedasticity. Each residual plot's MVI is computed using Equation
\ref{eq:mvi} with \(C = 10\). When \(\text{MVI} > 8\), the visual
patterns are notably strong and easily discernible by humans. In the
range \(6 < \text{MVI} < 8\), the visibility of the visual pattern
diminishes as MVI decreases. Conversely, when \(\text{MVI} < 6\), the
visual pattern tends to become relatively faint and challenging to
observe. Table \ref{tab:mvi} provides a summary of the MVI usage and it
is applicable to other linear regression models.

\begin{table}

\caption{\label{tab:mvi}Degree of model violations or the strength of the visual signals according to the Model Violations Index (MVI). The constant $C$ is set to be 10.}
\centering
\begin{tabular}[t]{lc}
\toprule
Degree of model violations & Range ($C$ = 10)\\
\midrule
Strong & $\text{MVI} > 8$\\
Moderate & $6 < \text{MVI} < 8$\\
Weak & $\text{MVI} < 6$\\
\bottomrule
\end{tabular}
\end{table}

\begin{figure}[!h]

{\centering \includegraphics[width=1\linewidth]{paper_files/figure-latex/poly-heter-index-1} 

}

\caption{Residual plots generated from fitted models exhibiting varying degrees of (A) non-linearity and (B) heteroskedasticity violations. The model violations index (MVI) is displayed atop each residual plot. The non-linearity patterns are relatively strong for $MVI > 8$, and relatively weak for $MVI < 6$, while the heteroskedasticity patterns are relatively strong for $MVI > 8$, and relatively weak for $MVI < 6$.}\label{fig:poly-heter-index}
\end{figure}

\section{Model Architecture}\label{sec-model-architecture}

\begin{figure}

{\centering \includegraphics[width=1\linewidth]{paper_files/figure-latex/cnn-diag-1} 

}

\caption{Diagram of the architecture of the optimized computer vision model. Numbers at the bottom of each box show the shape of the output of each layer. The band of each box drawn in a darker color indicates the use of the rectified linear unit activation function.  Yellow boxes are 2D convolutional layers, orange boxes are pooling layers, the grey box is the concatenation layer, and the purple boxes are dense layers.}\label{fig:cnn-diag}
\end{figure}

The architecture of the computer vision model \(f_{CV}\) is adapted from
the well-established VGG16 architecture \citep{simonyan2014very}. While
more recent architectures like ResNet \citep{he2016deep} and
DenseNet\citep{huang2017densely}, have achieved even greater
performance, VGG16 remains a solid choice for many applications due to
its simplicity and effectiveness. Our decision to use VGG16 aligns with
our goal of starting with a proven and straightforward model. Figure
\ref{fig:cnn-diag} provides a diagram of the architecture. More details
about the neural network layers used in this study are provided in the
Appendix.

The model begins with an input layer of shape
\(n \times h \times w \times 3\), capable of handling \(n\) RGB images.
This is followed by a grayscale conversion layer utilizing the luma
formula under the Rec. 601 standard \citep{series2011studio}, which
converts the color image to grayscale. Grayscale suffices for our task
since data points are plotted in black. We experiment with three
combinations of \(h\) and \(w\): \(32 \times 32\), \(64 \times 64\), and
\(128 \times 128\), aiming to achieve sufficiently high image resolution
for the problem at hand.

The processed image is used as the input for the first convolutional
block. The model comprises at most five consecutive convolutional
blocks, mirroring the original VGG16 architecture. Within each block,
there are two 2D convolutional layers followed by two activation layers,
respectively. Subsequently, a 2D max-pooling layer follows the second
activation layer. The 2D convolutional layer convolves the input with a
fixed number of \(3 \times 3\) convolution filters, while the 2D
max-pooling layer downsamples the input along its spatial dimensions by
taking the maximum value over a \(2 \times 2\) window for each channel
of the input. The activation layer employs the rectified linear unit
(ReLU) activation function, a standard practice in deep learning, which
introduces a non-linear transformation of the output of the 2D
convolutional layer. Additionally, to regularize training, a batch
normalization layer is added after each 2D convolutional layer and
before the activation layer. Finally, a dropout layer is appended at the
end of each convolutional block to randomly set some inputs to zero
during training, further aiding in regularization.

The output of the last convolutional block is summarized by either a
global max-pooling layer or a global average-pooling layer, resulting in
a two-dimensional tensor. To enrich predictions with additional
information beyond visual features, this tensor is concatenated with an
additional \(n \times 5\) tensor, which contains the ``Monotonic'',
``Sparse'', ``Splines'', and ``Striped'' measures computed using the
\texttt{cassowaryr} R package \citep{mason2022cassowaryr} , along with
the number of observations for \(n\) residual plots. These measures were
selected for their reliability and efficiency, as other scagnostics
occasionally caused R process crashes (\(\sim 5\%\)) during data
preparation due to a bug in the \texttt{interp} package
\citep{Albrecht2023interp}. Although this bug was later fixed at our
request, the fix came too late to re-train the model. Moreover, their
high computational cost makes them unsuitable for fast inference.

The concatenated tensor is then fed into the final prediction block.
This block consists of two fully-connected layers. The first layer
contains at least \(128\) units, followed by a dropout layer. A batch
normalization layer is inserted between the fully-connected layer and
the dropout layer for regularization purposes. The second
fully-connected layer consists of only one unit, serving as the output
of the model.

The model weights \(\boldsymbol{\theta}\) were randomly initialized
using the Glorot Uniform method \citep{glorot2010understanding} and they
were optimized by the Adam optimizer \citep{kingma2014adam} with the
mean square error loss function

\[\hat{\boldsymbol{\theta}} = \underset{\boldsymbol{\theta}}{\text{arg min}}\frac{1}{n_{\text{train}}}\sum_{i=1}^{n_{\text{train}}}(D_i - f_{\boldsymbol{\theta}}(V_i, S_i))^2,\]

\noindent where \(n_{\text{train}}\) is the number of training samples,
\(V_i\) is the \(i\)-th residual plot and \(S_i\) is the auxiliary
information about the \(i\)-th residual plot including four scagnostics
and the number of observations.

\section{Data Generation and Model
Training}\label{sec-model-data-generation}

To enable supervised training of computer vision models for detecting
violations in linear regression, we generated a synthetic dataset of
80,000 training and 8,000 test residual plots. This simulation-based
approach provided full control over the data-generating process,
allowing precise computation of \(D\), cost-effective data scaling, and
the ability to model diverse visual patterns of model violations.

The simulated data incorporated three common types of residual
departures: non-linearity, heteroskedasticity, and non-normality. These
were introduced by fitting a standard simple linear regression model to
data generated from a more flexible model based on Hermite polynomials
\citep{hermite1864nouveau}, multiple predictors, and varying
distributions in both predictors and error terms. A comprehensive grid
of parameter combinations was explored to generate a wide range of
residual patterns. Importantly, the simulation included scenarios with
multiple violations occurring simultaneously. To ensure uniform coverage
across the difficulty scale of the target variable \(D\), a bucket
sampling scheme was used to create a balanced dataset. Samples were
iteratively simulated and accepted into one of 50 buckets, each
representing a distinct range of \(D\).

Model training was conducted on the MASSIVE M3 high-performance
computing platform \citep{goscinski2014multi} using TensorFlow
\citep{abadi2016tensorflow} and Keras \citep{chollet2015keras}.
Hyperparameters were optimized via Bayesian tuning with KerasTuner
\citep{omalley2019kerastuner}, minimizing validation RMSE across 100
trials. The tuning process included early stopping and considered
dropout rate, batch normalization, input resolution, and auxiliary
inputs.

Further details, including mathematical formulations of the simulation
model, parameter specifications, sampling scheme, example residual plots
and hyperparameter tuning configuration, are provided in the Appendix.

\section{Results}\label{sec-model-results}

\subsection{Model Performance}\label{model-performance}

The test performance for the optimized models with three different input
sizes are summarized in Table \ref{tab:performance}. Among these models,
the \(32 \times 32\) model consistently exhibited the best test
performance. The mean absolute error of the \(32 \times 32\) model
indicated that the difference between \(\hat{D}\) and \(D\) was
approximately \(0.43\) on the test set, a negligible deviation
considering the normal range of \(D\) typically falls between \(0\) and
\(7\). The high \(R^2\) values also suggested that the predictions were
largely linearly correlated with the target.

Figure \ref{fig:model-performance} presents a hexagonal heatmap for
\(D - \hat{D}\) versus \(D\). The brown smoothing curves, fitted by
generalized additive models \citep{hastie2017generalized}, demonstrate
that all the optimized models perform admirably on the test sets when
\(1.5 < D < 6\), where no structural issues are noticeable. However,
over-predictions occurred when \(D < 1.5\), while under-predictions
occurred predominantly when \(\hat{D} > 6\).

For input images representing null plots where \(D = 0\), it was
expected that the models will over-predict the distance, as explained in
Section \ref{sec-model-lineup-evaluation}. However, it can not explain
the under-prediction issue. Therefore, we analysed the relationship
between residuals and all the factors involved in the data generating
process. We found that most issues actually arose from non-linearity
problems and the presence of a second predictor in the regression model
as illustrated in Figure \ref{fig:over-under}. When the variance for the
error distribution was small, the optimized model tended to
under-predict the distance. Conversely, when the error distribution had
a large variance, the model tended to over-predict the distance.

Since most of the deviation stemmed from the presence of non-linearity
violations, to further investigate this, we split the test set based on
violation types and re-evaluated the performance, as detailed in Table
\ref{tab:performance-sub}. It was evident that metrics for null plots
were notably worse compared to other categories. Furthermore, residual
plots solely exhibiting non-normality issues were the easiest to
predict, with very low test root mean square error (RMSE) at around
\(0.3\). Residual plots with non-linearity issues were more challenging
to assess than those with heteroskedasticity or non-normality issues.
When multiple violations were introduced to a residual plot, the
performance metrics typically lay between the metrics for each
individual violation.

Based on the model performance metrics, we chose to use the
best-performing model evaluated on the test set, namely the
\(32 \times 32\) model, for the subsequent analysis.

\begin{table}

\caption{\label{tab:performance}The test performance of three optimized models with different input sizes.}
\centering
\begin{tabular}[t]{lrrrr}
\toprule
 & RMSE & $R^2$ & MAE & Huber loss\\
\midrule
$32 \times 32$ & 0.660 & 0.901 & 0.434 & 0.18\\
$64 \times 64$ & 0.674 & 0.897 & 0.438 & 0.19\\
$128 \times 128$ & 0.692 & 0.892 & 0.460 & 0.20\\
\bottomrule
\end{tabular}
\end{table}

\begin{figure}[!h]

{\centering \includegraphics[width=1\linewidth]{paper_files/figure-latex/model-performance-1} 

}

\caption{Hexagonal heatmap for difference in $D$ and $\hat{D}$ vs $D$ on test data for three optimized models with different input sizes. The brown lines are smoothing curves produced by fitting generalized additive models. The area over the zero line in light yellow indicates under-prediction, and the area under the zero line in light green indicates over-prediction.}\label{fig:model-performance}
\end{figure}

\begin{figure}[!h]

{\centering \includegraphics[width=1\linewidth]{paper_files/figure-latex/over-under-1} 

}

\caption{Scatter plots for difference in $D$ and $\hat{D}$ vs $\sigma$ on test data for the $32 \times 32$ optimized model. The data is grouped by whether the regression has only non-linearity violation, and whether it includes a second predictor in the regression formula. The brown lines are smoothing curves produced by fitting generalized additive models. The area over the zero line in light yellow indicates under-prediction, and the area under the zero line in light green indicates over-prediction.}\label{fig:over-under}
\end{figure}

\begin{table}

\caption{\label{tab:performance-sub}The training and test performance of the $32 \times 32$ model presented with different model violations.}
\centering
\begin{tabular}[t]{lrr}
\toprule
Violations & \#samples & RMSE\\
\midrule
no violations & 155 & 1.267\\
non-linearity & 2218 & 0.787\\
heteroskedasticity & 1067 & 0.602\\
non-linearity + heteroskedasticity & 985 & 0.751\\
non-normality & 1111 & 0.320\\
non-linearity + non-normality & 928 & 0.600\\
heteroskedasticity + non-normality & 819 & 0.489\\
non-linearity + heteroskedasticity + non-normality & 717 & 0.620\\
\bottomrule
\end{tabular}
\end{table}

\subsection{Comparison with Human Visual Inference and Conventional
Tests}\label{comparison-with-human-visual-inference-and-conventional-tests}

\subsubsection{Overview of the Human Subject
Experiment}\label{overview-of-the-human-subject-experiment}

In order to check the validity of the proposed computer vision model,
residual plots presented in the human subject experiment conducted by
\citet{li2024plot} will be assessed.

This study has collected 7,974 human responses to 1,152 lineups. Each
lineup contains one randomly placed true residual plot and 19 null
plots. Among the 1,152 lineups, 24 are attention check lineups in which
the visual patterns are designed to be extremely obvious and very
different from the corresponding to null plots, 36 are null lineups
where all the lineups consist of only null plots, 279 are lineups with
uniform predictor distribution evaluated by 11 participants, and the
remaining 813 are lineups with discrete, skewed or normal predictor
distribution evaluated by 5 participants. Attention check lineups and
null lineups will not be assessed in the following analysis.

In \citet{li2024plot}, the residual plots are simulated from a data
generating process that corresponds to a special case of the synthetic
data model used in this study. A key feature of the design is that model
violations are introduced independently, meaning non-linearity and
heteroskedasticity do not co-exist within a single lineup but are
instead assigned uniformly across different lineups. Moreover, the
experimental design does not account for non-normality or multiple
predictors.

\subsubsection{Model Performance on the Human-evaluated
Data}\label{model-performance-on-the-human-evaluated-data}

\begin{table}

\caption{\label{tab:experiment-performance}The performance of the $32 \times 32$ model on the data used in the human subject experiment.}
\centering
\begin{tabular}[t]{lrrrr}
\toprule
Violation & RMSE & $R^2$ & MAE & Huber loss\\
\midrule
heteroskedasticity & 0.721 & 0.852 & 0.553 & 0.235\\
non-linearity & 0.738 & 0.770 & 0.566 & 0.246\\
\bottomrule
\end{tabular}
\end{table}

For each lineup used in \citet{li2024plot}, there is one true residual
plot and 19 null plots. While the distance \(D\) for the true residual
plot depends on the underlying data generating process, the distance
\(D\) for the null plots is zero. We have used our optimized computer
vision model to estimate distance for both the true residual plots and
the null plots. To have a fair comparison, \(H_0\) will be rejected if
the true residual plot has the greatest estimated distance among all
plots in a lineup. Additionally, the appropriate conventional tests
including the Ramsey Regression Equation Specification Error Test
(RESET) \citep{ramsey1969tests} for non-linearity and the Breusch-Pagan
test \citep{breusch1979simple} for heteroskedasticity were applied on
the same data for comparison.

The performance metrics of \(\hat{D}\) for true residual plots are
outlined in Table \ref{tab:experiment-performance}. It is notable that
all performance metrics are slightly worse than those evaluated on the
test data. Nevertheless, the mean absolute error remains at a low level,
and the linear correlation between the prediction and the true value
remains very high. Consistent with results in Table
\ref{tab:performance-sub}, lineups with non-linearity issues are more
challenging to predict than those with heteroskedasticity issues.

Table \ref{tab:human-conv-table} provides a summary of the agreement
between decisions made by the computer vision model and conventional
tests. The agreement rates between conventional tests and the computer
vision model are 85.95\% and 79.69\% for residual plots containing
heteroskedasticity and non-linearity patterns, respectively. These
figures are higher than those calculated for visual tests conducted by
human, indicating that the computer vision model exhibits behavior more
akin to the best available conventional tests. However, Figure
\ref{fig:conv-mosaic} shows that the computer vision model does not
always reject when the conventional tests reject. And a small number of
plots will be rejected by computer vision model but not by conventional
tests. This suggests that conventional tests are more sensitive than the
computer vision model.

Figure \ref{fig:pcp} further illustrates the decisions made by visual
tests conducted by human, computer vision models, and conventional
tests, using a parallel coordinate plots. It can be observed that all
three tests will agree with each other for around 50\% of the cases.
When visual tests conducted by human do not reject, there are
substantial amount of cases where computer vision model also do not
reject but conventional tests reject. There are much fewer cases that do
not reject by visual tests and conventional tests, but is rejected by
computer vision models. This indicates computer vision model can behave
like visual tests conducted by human better than conventional tests.
Moreover, there are great proportion of cases where visual tests
conducted by human is the only test who does not reject.

When plotting the decision against the distance, as illustrated in
Figure \ref{fig:power}, several notable observations emerge. Firstly,
compared to conventional tests, the computer vision model tends to have
fewer rejected cases when \(D < 2\) and fewer non-rejected cases when
\(2< D < 4\). This suggests tests based on the computer vision model are
less sensitive to small deviations from model assumptions than
conventional tests but more sensitive to moderate deviations.
Additionally, visual tests demonstrate the lowest sensitivity to
residual plots with small distances where not many residual plots are
rejected when \(D < 2\). Similarly, for large distances where \(D > 4\),
almost all residual plots are rejected by the computer vision model and
conventional tests, but for visual tests conducted by humans, the
threshold is higher with \(D > 5\).

In Figure \ref{fig:power}, rejection decisions are fitted by logistic
regression models with no intercept terms and an offset equals to
\(\text{log}(0.05/0.95)\). The fitted curves for the computer vision
model fall between those of conventional tests and visual tests for both
non-linearity and heteroskedasticity, which means there is still
potential to refine the computer vision model to better align its
behavior with visual tests conducted by humans.

In the experiment conducted in \citet{li2024plot}, participants were
allowed to make multiple selections for a lineup. The weighted detection
rate was computed by assigning weights to each detection. If the
participant selected zero plots, a weight of 0.05 was assigned;
otherwise, if the true residual plot was detected, the weight was 1
divided by the number of selections. This weighted detection rate allow
us to assess the quality of the distance measure purposed in this paper,
by using the \(\delta\)-difference statistic. The \(\delta\)-difference
is originally defined by \citet{chowdhury2018measuring}, is given by

\[
\delta = \bar{d}_{\text{true}} - \underset{j}{\text{max}}\left(\bar{d}_{\text{null}}^{(j)}\right) \quad \text{for}~j = 1,...,m-1,
\]

where \(\bar{d}_{\text{null}}^{(j)}\) is the mean distance between the
\(j\)-th null plot and the other null plots, \(\bar{d}_{\text{true}}\)
is the mean distance between the true residual plot and null plots, and
\(m\) is the number of plots in a lineup. These mean distances are used
because, as noted by \citet{chowdhury2018measuring}, the distances can
vary depending on which data plot is used for comparison. For instance,
with three null plots, A, B and C, the distance between A and B may
differ from the distance between A and C. To obtain a consistent
distance for null plot A, averaging is necessary. However, this approach
is not applicable to the distance proposed in this paper, as we only
compare the residual plot against a theoretically good residual plot.
Consequently, the statistic must be adjusted to evaluate our distance
measure effectively.

One important aspect that the \(\delta\)-difference was designed to
capture is the empirical distribution of distances for null plot. If we
were to replace the mean distances \(\bar{d}_{\text{null}}^{(j)}\)
directly with \(D_{\text{null}}^{(j)}\), the distance of the \(j\)-th
null plot, the resulting distribution would be degenerate, since
\(D_{null}\) equals zero by definition. Additionally, \(D\) can not be
derived from an image, meaning it falls outside the scope of the
distances considered by \citet{chowdhury2018measuring}. Instead, the
focus should be on the empirical distribution of \(\hat{D}\), as it
influences decision-making. Therefore, the adjusted \(\delta\)-different
is defined as

\[
\delta_{\text{adj}} = \hat{D} - \underset{j}{\text{max}}\left(\hat{D}_{\text{null}}^{(j)}\right) \quad \text{for}~j = 1,...,m-1,
\]

\noindent where \(\hat{D}_{\text{null}}^{(j)}\) is the estimated
distance for the \(j\)-th null plot, and \(m\) is the number of plots in
a lineup.

Figure \ref{fig:delta} displays the scatter plot of the weighted
detection rate vs the adjusted \(\delta\)-difference. It indicates that
the weighted detection rate increases as the adjusted
\(\delta\)-difference increases, particularly when the adjusted
\(\delta\)-difference is greater than zero. A negative adjusted
\(\delta\)-difference suggests that there is at least one null plot in
the lineup with a stronger visual signal than the true residual plot. In
some instances, the weighted detection rate is close to one, yet the
adjusted \(\delta\)-difference is negative. This discrepancy implies
that the distance measure, or the estimated distance, may not perfectly
reflect actual human behavior.

\begin{table}

\caption{\label{tab:human-conv-table}Summary of the comparison of decisions made by computer vision model with decisions made by conventional tests and visual tests conducted by human.}
\centering
\begin{tabular}[t]{lrrr}
\toprule
Violations & \#Samples & \#Agreements & Agreement rate\\
\midrule
\addlinespace[0.3em]
\multicolumn{4}{l}{\textbf{Compared with conventional tests}}\\
\hspace{1em}heteroskedasticity & 540 & 464 & 0.8593\\
\hspace{1em}non-linearity & 576 & 459 & 0.7969\\
\addlinespace[0.3em]
\multicolumn{4}{l}{\textbf{Compared with visual tests conducted by human}}\\
\hspace{1em}heteroskedasticity & 540 & 367 & 0.6796\\
\hspace{1em}non-linearity & 576 & 385 & 0.6684\\
\bottomrule
\end{tabular}
\end{table}

\begin{figure}[!h]

{\centering \includegraphics[width=1\linewidth]{paper_files/figure-latex/conv-mosaic-1} 

}

\caption{Rejection rate ($p$-value $\leq0.05$) of computer vision models conditional on conventional tests on non-linearity (left) and heteroskedasticity (right) lineups displayed using a mosaic plot. When the conventional test fails to reject, the computer vision mostly fails to reject the same plot as well as indicated by the height of the top right yellow rectangle, but there are non negliable amount of plots where the conventional test rejects but the computer vision model fails to reject as indicated by the width of the top left yellow rectangle.}\label{fig:conv-mosaic}
\end{figure}

\begin{figure}[!h]

{\centering \includegraphics[width=1\linewidth]{paper_files/figure-latex/power-1} 

}

\caption{Comparison of power of visual tests, conventional tests and the computer vision model. Marks along the x-axis at the bottom of the plot represent rejections made by each type of test. Marks at the top of the plot represent acceptances. Power curves are fitted by logistic regression models with no intercept but an offset equals to $\text{log}(0.05/0.95)$.}\label{fig:power}
\end{figure}

\begin{figure}

{\centering \includegraphics[width=1\linewidth]{paper_files/figure-latex/pcp-1} 

}

\caption{Parallel coordinate plots of decisions made by computer vision model, conventional tests and visual tests made by human.}\label{fig:pcp}
\end{figure}

\begin{figure}[!h]

{\centering \includegraphics[width=1\linewidth]{paper_files/figure-latex/delta-1} 

}

\caption{A weighted detection rate vs adjusted $\delta$-difference plot. The brown line is smoothing curve produced by fitting generalized additive models.}\label{fig:delta}
\end{figure}

\section{Examples}\label{sec-examples}

In this section, we present the performance of trained computer vision
model on three example datasets. These include the dataset associated
with the residual plot displaying a ``left-triangle'' shape, as
displayed in Figure \ref{fig:false-finding}, along with the Boston
housing dataset \citep{harrison1978hedonic}, and the ``dino'' datasets
from the \texttt{datasauRus} R package \citep{datasaurus}.

The first example illustrates a scenario where both the computer vision
model and human visual inspection successfully avoid rejecting \(H_0\)
when \(H_0\) is true, contrary to conventional tests. This underscores
the necessity of visually examining the residual plot.

In the second example, we encounter a more pronounced violation of the
model, resulting in rejection of \(H_0\) by all three tests. This
highlights the practicality of the computer vision model, particularly
for less intricate tasks.

The third example presents a situation where the model deviation is
non-typical. Here, the computer vision model and human visual inspection
reject \(H_0\), whereas some commonly used conventional tests do not.
This emphasizes the benefits of visual inspection and the unique
advantage of the computer vision model, which, like humans, makes
decisions based on visual discoveries.

\subsection{Left-triangle}\label{left-triangle}

In Section \ref{sec-model-introduction}, we presented an example
residual plot showcased in Figure \ref{fig:false-finding}, illustrating
how humans might misinterpret the ``left-triangle'' shape as indicative
of heteroskedasticity. Additionally, the Breusch-Pagan test yielded a
rejection with a \(p\)-value of 0.046, despite the residuals originating
from a correctly specified model. Figure \ref{fig:false-lineup} offers a
lineup for this fitted model, showcasing various degrees of
``left-triangle'' shape across all residual plots. This phenomenon is
evidently caused by the skewed distribution of the fitted values.
Notably, if the residual plot were evaluated through a visual test, it
would not be rejected since the true residual plot positioned at 10 can
not be distinguished from the others.

Figure \ref{fig:false-check} presents the results of the assessment by
the computer vision model. Notably, the observed visual signal strength
is considerably lower than the 95\% sample quantile of the null
distribution. Moreover, the bootstrapped distribution suggests that it
is highly improbable for the fitted model to be misspecified as the
majority of bootstrapped fitted models will not be rejected. Thus, for
this particular fitted model, both the visual test and the computer
vision model will not reject \(H_0\). However, the Breusch-Pagan test
will reject \(H_0\) because it can not effectively utilize information
from null plots.

The attention map at Figure \ref{fig:false-check}B suggests that the
estimation is highly influenced by the top-right and bottom-right part
of the residual plot, as it forms two vertices of the triangular shape.
A principal component analysis (PCA) is also performed on the output of
the global pooling layer of the computer vision model. As mentioned in
\citet{simonyan2014very}, a computer vision model built upon the
convolutional blocks can be viewed as a feature extractor. For the
\(32 \times 32\) model, there are 256 features outputted from the global
pooling layer, which can be further used for different visual tasks not
limited to distance prediction. To see if these features can be
effectively used for distinguishing null plots and true residual plot,
we linearly project them into the first and second principal components
space as shown in Figure \ref{fig:false-check}D. It can be observed that
because the bootstrapped plots are mostly similar to the null plots, the
points drawn in different colors are mixed together. The true residual
plot is also covered by both the cluster of null plots and cluster of
bootstrapped plots. This accurately reflects our understanding of Figure
\ref{fig:false-lineup}.

\begin{figure}[!h]

{\centering \includegraphics[width=1\linewidth]{paper_files/figure-latex/false-check-1} 

}

\caption{A summary of the residual plot assessment evaluted on 200 null plots and 200 bootstrapped plots. (A) The true residual plot exhibiting a "left-triangle" shape. (B) The attention map produced by computing the gradient of the output with respect to the greyscale input.  (C) The density plot of estimated distance for null plots and bootstrapped plots. The green area indicates the distribution of estimated distances for bootstrapped plots, while the yellow area represents the distribution of estimated distances for null plots. The fitted model will not be rejected since $\hat{D} < Q_{null}(0.95)$. (D) plot of first two principal components of features extracted from the global pooling layer of the computer vision model.  }\label{fig:false-check}
\end{figure}

\begin{figure}[!h]

{\centering \includegraphics[width=1\linewidth]{paper_files/figure-latex/false-lineup-1} 

}

\caption{A lineup of residual plots displaying "left-triangle" visual patterns. The true residual plot occupies position 10, yet there are no discernible visual patterns that distinguish it from the other plots.}\label{fig:false-lineup}
\end{figure}

\subsection{Boston Housing}\label{boston-housing}

The Boston housing dataset, originally published by
\citet{harrison1978hedonic}, offers insights into housing in the Boston,
Massachusetts area. For illustration purposes, we utilize a reduced
version from Kaggle, comprising 489 rows and 4 columns: average number
of rooms per dwelling (RM), percentage of lower status of the population
(LSTAT), pupil-teacher ratio by town (PTRATIO), and median value of
owner-occupied homes in \$1000's (MEDV). In our analysis, MEDV will
serve as the response variable, while the other columns will function as
predictors in a linear regression model. Our primary focus is to detect
non-linearity, because the relationships between RM and MEDV or LSTAT
and MEDV are non-linear.

Figure \ref{fig:boston-check} displays the residual plot and the
assessment conducted by the computer vision model. A clear non-linearity
pattern resembling a ``U'' shape is shown in the plot A. Furthermore,
the RESET test yields a very small \(p\)-value. The estimated distance
\(\hat{D}\) significantly exceeds \(Q_{null}(0.95)\), leading to
rejection of \(H_0\). The bootstrapped distribution also suggests that
almost all the bootstrapped fitted models will be rejected, indicating
that the fitted model is unlikely to be correctly specified. The
attention map in plot B suggests the center of the image has higher
leverage than other areas, and it is the turning point of the ``U''
shape. The PCA provided in plot D shows two distinct clusters of data
points, further underling the visual differences between bootstrapped
plots and null plots. This coincides the findings from Figure
\ref{fig:boston-lineup}, where the true plot exhibiting a ``U'' shape is
visually distinctive from null plots. If a visual test is conducted by
human, \(H_0\) will also be rejected.

\begin{figure}[!h]

{\centering \includegraphics[width=1\linewidth]{paper_files/figure-latex/boston-check-1} 

}

\caption{A summary of the residual plot assessment for the Boston housing fitted model evaluted on 200 null plots and 200 bootstrapped plots. (A) The true residual plot exhibiting a "U" shape. (B) The attention map produced by computing the gradient of the output with respect to the greyscale input.  (C) The density plot of estimated distance for null plots and bootstrapped plots. The blue area indicates the distribution of estimated distances for bootstrapped plots, while the yellow area represents the distribution of estimated distances for null plots. The fitted model will be rejected since $\hat{D} \geq Q_{null}(0.95)$. (D) plot of first two principal components of features extracted from the global pooling layer of the computer vision model. }\label{fig:boston-check}
\end{figure}

\begin{figure}[!h]

{\centering \includegraphics[width=1\linewidth]{paper_files/figure-latex/boston-lineup-1} 

}

\caption{A lineup of residual plots for the Boston housing fitted model. The true residual plot is at position 7. It can be easily identified as the most different plot.}\label{fig:boston-lineup}
\end{figure}

\subsection{DatasauRus}\label{datasaurus}

The computer vision model possesses the capability to detect not only
typical issues like non-linearity, heteroskedasticity, and non-normality
but also artifact visual patterns resembling real-world objects, as long
as they do not appear in null plots. These visual patterns can be
challenging to categorize in terms of model violations. Therefore, we
will employ the RESET test, the Breusch-Pagan test, and the Shapiro-Wilk
test \citep{shapiro1965analysis} for comparison.

The ``dino'' dataset within the \texttt{datasauRus} R package
exemplifies this scenario. With only two columns, x and y, fitting a
regression model to this data yields a residual plot resembling a
``dinosaur'', as displayed in Figure \ref{fig:dino-check}A.
Unsurprisingly, this distinct residual plot stands out in a lineup, as
shown in Figure \ref{fig:dino-lineup}. A visual test conducted by humans
would undoubtedly reject \(H_0\).

According to the residual plot assessment by the computer vision model,
\(\hat{D}\) exceeds \(Q_{null}(0.95)\), warranting a rejection of
\(H_0\). Additionally, most of the bootstrapped fitted models will be
rejected, indicating an misspecified model. However, both the RESET test
and the Breusch-Pagan test yield \(p\)-values greater than 0.3, leading
to a non-rejection of \(H_0\). Only the Shapiro-Wilk test rejects the
normality assumption with a small \(p\)-value.

More importantly, the attention map in Figure \ref{fig:dino-check}B
clearly exhibits a ``dinosaur'' shape, strongly suggesting that the
distance prediction is based on human-perceptible visual patterns. The
computer vision model effectively captures the contour or outline of the
embedded shape, similar to how humans interpret residual plots.
Additionally, the PCA in Figure \ref{fig:dino-check}D demonstrates that
the cluster of bootstrapped plots is positioned at the corner of the
cluster of null plots.

In practice, without accessing the residual plot, it would be
challenging to identify the artificial pattern of the residuals.
Moreover, conducting a normality test for a fitted regression model is
not always standard practice among analysts. Even when performed,
violating the normality assumption is sometimes deemed acceptable,
especially considering the application of quasi-maximum likelihood
estimation in linear regression. This example underscores the importance
of evaluating residual plots and highlights how the proposed computer
vision model can facilitate this process.

\begin{figure}[!h]

{\centering \includegraphics[width=1\linewidth]{paper_files/figure-latex/dino-check-1} 

}

\caption{A summary of the residual plot assessment for the datasauRus fitted model evaluated on 200 null plots and 200 bootstrapped plots. (A) The residual plot exhibits a "dinosaur" shape. (B) The attention map produced by computing the gradient of the output with respect to the greyscale input.  (C) The density plot of estimated distance for null plots and bootstrapped plots. The blue area indicates the distribution of estimated distances for bootstrapped plots, while the yellow area represents the distribution of estimated distances for null plots. The fitted model will be rejected since $\hat{D} \geq Q_{null}(0.95)$. (D) plot of first two principal components of features extracted from the global pooling layer of the computer vision model.}\label{fig:dino-check}
\end{figure}

\begin{figure}[!h]

{\centering \includegraphics[width=1\linewidth]{paper_files/figure-latex/dino-lineup-1} 

}

\caption{A lineup of residual plots for the fitted model on the "dinosaur" dataset. The true residual plot is at position 17. It can be easily identified as the most different plot as the visual pattern is extremely artificial.}\label{fig:dino-lineup}
\end{figure}

\section{Limitations and Future Work}\label{limitations-and-future-work}

Despite the computer vision model performing well with general cases
under the synthetic data generation scheme and the three examples used
in this paper, this study has several limitations that could guide
future work.

The proposed distance measure assumes that the true model is a classical
normal linear regression model, which can be restrictive. Although this
paper does not address the relaxation of this assumption, there are
potential methods to evaluate other types of regression models. The most
comprehensive approach would be to define a distance measure for each
different class of regression model and then train the computer vision
model following the methodology described in this paper. To accelerate
training, one could use the convolutional blocks of our trained model as
a feature extractor and perform transfer learning on top of it, as these
blocks effectively capture shapes in residual plots. Another approach
would be to transform the residuals so they are roughly normally
distributed and have constant variance. If only raw residuals are used,
the distance-based statistical testing compares the difference in
distance to a classical normal linear regression model for the true plot
and null plots. This comparison is meaningful only if the difference can
be identified by the distance measure proposed in this paper.

There are other types of residual plots commonly used in diagnostics,
such as residuals vs.~predictor and quantile-quantile plots. In this
study, we focused on the most commonly used residual plot as a starting
point for exploring the new field of automated visual inference.
Similarly, we did not explore other, more sophisticated computer vision
model architectures and specifications for the same reason. While the
performance of the computer vision model is acceptable, there is still
room for improvement to achieve behavior more closely resembling that of
humans interpreting residual plots. This may require external survey
data or human subject experiment data to understand the fundamental
differences between our implementation and human evaluation.

\section{Conclusions}\label{conclusions}

In this paper, we have introduced a distance measure based on
Kullback-Leibler divergence to quantify the disparity between the
residual distribution of a fitted classical normal linear regression
model and the reference residual distribution assumed under correct
model specification. This distance measure effectively captures the
magnitude of model violations in misspecified models. We propose a
computer vision model to estimate this distance, utilizing the residual
plot of the fitted model as input. The resulting estimated distance
serves as the foundation for constructing a single Model Violations
Index (MVI), facilitating the quantification of various model
violations.

Moreover, the estimated distance enables the development of a formal
statistical testing procedure by evaluating a large number of null plots
generated from the fitted model. Additionally, employing bootstrapping
techniques and refitting the regression model allows us to ascertain how
frequently the fitted model is considered misspecified if data were
repeatedly obtained from the same data generating process.

The trained computer vision model demonstrates strong performance on
both the training and test sets, although it exhibits slightly lower
performance on residual plots with non-linearity visual patterns
compared to other types of violations. The statistical tests relying on
the estimated distance predicted by the computer vision model exhibit
lower sensitivity compared to conventional tests but higher sensitivity
compared to visual tests conducted by humans. While the estimated
distance generally mirrors the strength of the visual signal perceived
by humans, there remains scope for further improvement in its
performance.

Several examples are provided to showcase the effectiveness of the
proposed method across different scenarios, emphasizing the similarity
between visual tests and distance-based tests. Overall, both visual
tests and distance-based tests can be viewed as ensemble of tests,
aiming to assess any violations of model assumptions collectively. In
contrast, individual residual diagnostic tests such as the RESET test
and the Breusch-Pagan test only evaluate specific violations of model
assumptions. In practice, selecting an appropriate set of statistical
tests for regression diagnostics can be challenging, particularly given
the necessity of adjusting the significance level for each test.

Our method holds significant value as it helps alleviate a portion of
analysts' workload associated with assessing residual plots. While we
recommend analysts to continue reading residual plots whenever feasible,
as they offer invaluable insights, our approach serves as a valuable
tool for automating the diagnostic process or for supplementary purposes
when needed.

\section*{Acknowledgement}\label{acknowledgement}
\addcontentsline{toc}{section}{Acknowledgement}

These \texttt{R} packages were used for the work: \texttt{tidyverse}
\citep{tidyverse}, \texttt{lmtest} \citep{lmtest}, \texttt{mpoly}
\citep{mpoly}, \texttt{ggmosaic} \citep{ggmosaic}, \texttt{kableExtra}
\citep{kableextra}, \texttt{patchwork} \citep{patchwork},
\texttt{rcartocolor} \citep{rcartocolor}, \texttt{glue} \citep{glue},
\texttt{ggpcp} \citep{ggpcp}, \texttt{here} \citep{here},
\texttt{magick} \citep{magick}, \texttt{yardstick} \citep{yardstick} and
\texttt{reticulate} \citep{reticulate}.

The article was created with R packages \texttt{rticles}
\citep{rticles}, \texttt{knitr} \citep{knitr} and \texttt{rmarkdown}
\citep{rmarkdown}. The project's GitHub repository
(\url{https://github.com/TengMCing/auto_residual_reading_paper})
contains all materials required to reproduce this article.

\bibliographystyle{tfcad}
\bibliography{bibliography.bib}





\end{document}
